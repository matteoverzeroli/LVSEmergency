\paragraph{API APRS.FI}
Nella figura \Fig\ref{fig:APIAPRS} è riportato un esempio di chiamata all'API fornita da APRS.FI e la successiva risposta ricevuta.
Affinché si possano ricevere i dati relativi ad una stazione è necessario definire i seguenti parametri:
\begin{itemize}
	\item \textit{name}: identificativo della stazione meteorologica;
	\item \textit{what}: definizione della tipologia della stazione. Nel nostro caso siamo interessati a stazioni meteorologiche identificate dal valore \textit{wx};
	\item \textit{apikey}: API key identificativo dell'utente che richiede i dati;
	\item \textit{format}: formato della risposta ricevuta (può essere scelto json o xml);
\end{itemize}
La risposta conterrà, se la richiesta è andata a buon fine, i seguenti campi:
\begin{itemize}
	\item \textit{command}: tipo di richiesta http;
	\item \textit{result}: indica se la richiesta è andata a buon fine;
	\item \textit{found}: numero di stazione trovate;
	\item \textit{name}: identificativo della richiesta;
	\item \textit{time}: istante di acquisizione della misurazione corrente, nel formato \textit{Unix time};
	\item \textit{temp}: temperatura in gradi Celsius;
	\item \textit{pressure}: pressione in mbar;
	\item \textit{humidity}: umidità relativa;
	\item \textit{wind\_direction}: direzione del vento in gradi;
	\item \textit{wind\_speed}: velocità del vento media in metri al secondo;
	\item \textit{wind\_gust}: raffiche di vento in metri al secondo;
	\item \textit{rain\_1h}: pioggia cumulata nell'ultima ora in millimetri;
	\item \textit{rain\_24h}: pioggia cumulata nell'ultimo giorno in millimetri;
	\item \textit{rain\_mn}: pioggia cumulata dalla mezzanotte in millimetri;
	\item \textit{luminosity}: luminosità rilevata;
\end{itemize}

Tutti i campi sono opzionali e dipendono dai sensori attivi sulla particolare stazione. Tutta la documentazione sulle API esposte dal sito APRS.FI sono disponibili \url{https://aprs.fi/page/api}.

\begin{figure}[h!]
	\centering
	\includegraphics[width=1\linewidth]{./Iterazione 3/ImageFiles/APIAPRSRequest}
	\lstinputlisting[language=C++]{./Iterazione 3/OtherFiles/APIAPRSReponse.json}
	\caption{Esempio chiamata API APRS.FI}
	\label{fig:APIAPRS}
\end{figure}


