\section{Testing}
\subsection{Analisi statica}
Per l'analisi statica del codice Java è stato utilizzato il tool STAN4J, integrato nel IDE Eclipse.

\subsection{Analisi dinamica}
Nell'iterazione 3 si sono testate tutte le API Rest implementate, utlizzando Postman (\Fig \ref{fig:RisultatiTestAPIIT3}). In particolare si sono testate le seguenti funzionalità:

\begin{itemize}
	\item AreaController:
	\begin{itemize}
		\item Visualizzazione degli ultimi dati meteo scaricati di un'area, il cui id è specificato nel oath della richiesta;
		\item Visualizzazione del più recente allarme relativo a nebbia o brina associata all'area il cui id è specificato nel path della richiesta;
	\end{itemize}

	\begin{figure}[h!]
		\centering
		\includegraphics[width=1\linewidth]{./Iterazione 3/ImageFiles/TestGetAprsData}
		\includegraphics[width=1\linewidth]{./Iterazione 3/ImageFiles/TestGetFrostOrFogAlarm}
		\caption{Risultati test API su Postman.}
		\label{fig:RisultatiTestAPIIT3}
	\end{figure}
\end{itemize}

\clearpage

\subsection{Unit Test}
\subsubsection{LvsEmergency Server}
\todo{da inserire}
\subsubsection{Client App}
\todo{forse da inserire}
\subsection{Documentazione API}

In questa sezione viene mostrata la documentazione relativa ad alcune API implementate nell'iterazione 3. \`E possibile visualizzare una collezione (creata con Postman) di tutte le API realizzate e testate sulla repository GitHub. In particolare, si riportano di seguito le più significative:
\begin{itemize}
	\item API per la visualizzazione degli ultimi dati meteorologici raccolti dalla stazione APRS associata all'area, il cui id è specificato nel path della richiesta;
	\item API per la visualizzazione del più recente allarme di nebbia o brina generato nell'area di interesse;
\end{itemize}

\begin{figure}[h!]
	\centering
	\includegraphics[width=1\linewidth]{./Iterazione 3/ImageFiles/GetAprsDataRequest}
	\lstinputlisting[language=json]{./Iterazione 3/OtherFiles/GetAprsDataResponse.json}
	\caption{Documentazione API Get Aprs data.}
	\label{fig:LoginAPI}
\end{figure}

\begin{figure}[h!]
	\centering
	\includegraphics[width=1\linewidth]{./Iterazione 3/ImageFiles/GetFrostOrFogAlarmRequest}
	\lstinputlisting[language=json]{./Iterazione 3/OtherFiles/GetFrostOrFogAlarmResponse.json}
	\caption{Documentazione API Get fog or frost alarm.}
	\label{fig:GetOneUserAPI}
\end{figure}
