\section{Manuale Utente}
Grazie alle scelte tecnologiche con cui sono stati implementati i diversi componenti, il prodotto software può essere utilizzato ovunque e sulla maggior parte dei sistemi operativi, sia desktop (Windows, MacOs, Linux), sia mobile (Android o iOs).\todo{Davide chiede: "Si scrive così iOs?"}
L'applicativo si compone di quattro macro componenti:
\begin{itemize}
	\item \textbf{Web Server Spring}: questa componente è in esecuzione su una istanza di Azure Sping Clound e di conseguenza è sempre accessibile dall'applicazione client;
	\item \textbf{Database MySql}: anch'esso in esecuzione su una istanza di Azure MySql;
	\item \textbf{Data Collector e Data Analyzer}: queste due applicazioni sono in esecuzione su un Rapsberry Pi 4;
	\item \textbf{Applicazione Client}: grazie al framework Qt, tale applicazione può essere compilata ed essere eseguita su una gran parte dei sistemi operativi. Infatti, l'unica cosa che deve fare l'utente è installare l'APK sul proprio telefono, oppure installare l'applicazione sul proprio computer. 
\end{itemize}

In questo modo, l'utente può accedere ed utilizzate l'applicazione ed i servizi offerti in qualunque momento. 
\\
In figura x viene mostrata la schermata di login dell'applicazione. L'utente deve inserire username e password che gli sono stati assegnati dal coordinatore. 
Dopo aver inserito le credenziali, gli verrà mostrata la dashboard (figura x) in cui è possibile osservare alcune informazioni sul proprio account ed una mappa che mostra la posizione dell'area di appartenenza.
Nel menù laterale è possibile scegliere quali servizi utilizzare (figura x): si possono visualizzare informazioni meteorologiche (figura x), allarmi (figura x) o informazioni sul proprio account e sul proprio team, compresi alcuni dati sui membri del team (figura x).

Un amministratore del sistema potrà invece scegliere dei servizi diversi da quelli di un utente normale (figura x). Infatti, tale figura ha il compito di inserire nuovi volontari inserendone i dati (figura x), creare nuove squadre (figura z) e cancellare degli utenti (figura z).

\todo{Da rileggere}

\section{Sviluppi futuri}
Non tutti i casi d'uso definiti sono stati implementati. Di seguito è riportata una tabella riassuntiva di quali casi d'uso sono stati implementati e quali no:

\begin{center}
	\begin{tabular}{|c|c|c|}
		\hline
		\textbf{Codice} & \textbf{Caso d'uso} & \textbf{Implementato} \\ \hline
		\multicolumn{3}{|c|}{Alta Priorità} \\ \hline
		\textbf{UC1} & Login & Sì\\ \hline
		\textbf{UC2} & Logout & Sì \\ \hline
		\textbf{UC3} & Visualizzazione informazioni account & Sì \\ \hline
		\textbf{UC4} & Visualizzazione informazioni squadra & Sì \\ \hline
		\textbf{UC15} & Inserimento utente & Sì \\ \hline
		\textbf{UC16} & Cancellazione utente & Sì\\ \hline
		\textbf{UC17} & Gestione squadre (creazione) & Sì \\ \hline
		\textbf{UC18} & Visualizzazione informazioni zona & Sì \\ \hline
		\multicolumn{3}{|c|}{Media Priorità} \\ \hline
		\textbf{UC6} & Segnalazione operatività & Sì\\ \hline
		\textbf{UC7} & Visualizzazione intervento di emergenza & No \\ \hline
		\textbf{UC8} & Visualizzazione intervento programmato & No \\ \hline
		\textbf{UC9} & Inserimento informazioni intervento & No \\ \hline
		\textbf{UC11} & Gestione intervento di emergenza & No \\ \hline
		\textbf{UC12} & Gestione intervento programmato & No \\ \hline
		\textbf{UC13} & Gestione report intervento di emergenza & No \\ \hline
		\textbf{UC14} & Gestione report intervento programmato & No \\ \hline
		\textbf{UC20} & Gestione informazioni relative alla zona & No \\ \hline
		\multicolumn{3}{|c|}{Bassa Priorità} \\ \hline
		\textbf{UC5} & Gestione reperibilità & No\\ \hline
		\textbf{UC10} & Visualizzazione posizione real-time & No\\ \hline
		\textbf{UC19} & Notifiche allarmi zona & No\\ \hline
	\end{tabular}
\end{center}
 