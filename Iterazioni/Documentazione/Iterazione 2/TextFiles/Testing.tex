\section{Testing}
\subsection{Analisi statica}
Per l'analisi statica del codice Java è stato utilizzato il tool STAN4J, integrato nel IDE Eclipse. Il report è riportato a fine capitolo.

\subsection{Analisi dinamica}
Nell'iterazione 2 sono state testate tutte le API REST implementate, utlizzando Postman (\Fig \ref{fig:RisultatiTestAPIIT2}). In particolare sono state testate le seguenti funzionalità:

\begin{itemize}
	\item UserController:
	\begin{itemize}
		\item Login con credenziali corrette e verifica che le informazioni dell'utente ricevute siano corrette;
		\item Login con credenziali errate;
		\item Visualizzazione di tutti gli utenti registrati nel sistema;
		\item Visualizzazione di un utente specifico e verifica della correttezza dei dati ritornati;
		\item Eliminazione utente;
		\item Inserimento di un nuovo utente e verifica che i dati dell'utente siano stati inseriti in modo corretto;
		\item Modifica utente e verifica che l'utente sia stato modificato correttamente;
	\end{itemize}
	\item TeamController:
	\begin{itemize}
		\item Visualizzazione di una squadra e verifica che le informazioni ricevute siano corrette;
		\item Visualizzazione di tutte le squadre inserite nel sistema;
		\item Eliminazione di una squadra;
		\item Inserimento di una nuova squadra;
		\item Assegnamento di un caposquadra a una squadra;
	\end{itemize}
	\item AreaController:
	\begin{itemize}
		\item Visualizzazione di tutte le aree inserite nel sistema;
	\end{itemize}

	\begin{figure}[h!]
		\centering
		\includegraphics[width=0.99\linewidth]{./Iterazione 2/ImageFiles/TestUserController1}
		\includegraphics[width=0.99\linewidth]{./Iterazione 2/ImageFiles/TestUserController2}
		\includegraphics[width=1\linewidth]{./Iterazione 2/ImageFiles/TestTeamController}
		\includegraphics[width=0.99\linewidth]{./Iterazione 2/ImageFiles/TestSetIdForeman}
		\includegraphics[width=1\linewidth]{./Iterazione 2/ImageFiles/TestAreaController}
		\caption{Risultati test API su Postman.}
		\label{fig:RisultatiTestAPIIT2}
	\end{figure}
\end{itemize}

\clearpage

\subsection{Unit Test}
\subsubsection{LvsEmergency Server}
In questa iterazione è stata testata la funzione \texttt{findByUsername(String username)} che permette di recuperare l'utente inserito nel database con lo \textit{username} specificato. Inoltre viene eseguito il test di avvio del framework Spring, inserito di default. Di seguito è riportato il codice del test.

\lstinputlisting[language=Java]{./Iterazione 2/OtherFiles/UserRepositoryTest.java}

Il risultato del test con JUnit ha confermato il corretto funzionamento della funzione.

\begin{figure}[h!]
	\centering
	\includegraphics[width=1\linewidth]{./Iterazione 2/ImageFiles/TestJUnit}
	\caption{Risultato test con JUnit.}
	\label{fig:RisultatiTestJunitIT2}
\end{figure}

\subsubsection{Client app}
Lato client è stata testata la corretta creazione delle classi \texttt{User}, \texttt{Team} e \texttt{Area} a partire da una stringa JSON. Per fare ciò, è stato utilizzato il framework \textit{Qt Test}. Per effettuare il parsing è stata utilizzata la libreria \textit{QJsonDocument}, fornita da Qt: si tratta di una classe che è in grado di analizzare un documento JSON con funzionalità di lettura e scrittura. Di seguito viene riportato il codice del test.
 
\lstinputlisting[language=C++]{./Iterazione 2/OtherFiles/testentityif.cpp}

Il risultato del test ha confermato il corretto funzionamento delle funzioni.

\begin{figure}[h!]
	\centering
	\includegraphics[width=1\linewidth]{./Iterazione 2/ImageFiles/TestQtTest}
	\caption{Risultato test con Qt Test.}
	\label{fig:RisultatiTestQTestIT2}
\end{figure}

\clearpage

\section{Documentazione API}

In questa sezione viene mostrata la documentazione relativa ad alcune API implementate nell'iterazione 2. \`E possibile visualizzare una collezione (creata con Postman) di tutte le API realizzate e testate sulla repository GitHub. In particolare, si riportano di seguito le più significative:
\begin{itemize}
	\item API per il login (nel header della richiesta sono state inserite correttamente le informazioni secondo il protocollo della Basic Authentication);
	\item API per la visualizzazione di un utente specifico, il cui id è inserito nel path della chiamata;
	\item API per la modifica di un utente, inserendo nel body della richiesta l'utente modificato;
	\item API per visualizzare un team, il cui id viene specificato nel path della richiesta;
	\item API per la creazione di un team, inserendo nel body della richiesta le informazioni del team;
	\item API per l'assegnamento di un caposquadra a una squadra;
	\item API per la visualizzazione di tutte le area inserite nel sistema;
\end{itemize}

\begin{figure}[h!]
	\centering
	\includegraphics[width=1\linewidth]{./Iterazione 2/ImageFiles/LoginCorrettoRequest}
	\lstinputlisting[language=json]{./Iterazione 2/OtherFiles/LoginCorrettoResponse.json}
	\caption{Documentazione API Login.}
	\label{fig:LoginAPI}
\end{figure}

\begin{figure}[h!]
	\centering
	\includegraphics[width=1\linewidth]{./Iterazione 2/ImageFiles/GetOneUser}
	\lstinputlisting[language=json]{./Iterazione 2/OtherFiles/GetOneUserResponse.json}
	\caption{Documentazione API Get one user.}
	\label{fig:GetOneUserAPI}
\end{figure}

\begin{figure}[h!]
	\centering
	\includegraphics[width=1\linewidth]{./Iterazione 2/ImageFiles/ModifyAUser}
	Body:
	\lstinputlisting[language=json]{./Iterazione 2/OtherFiles/ModifyAUserResponse.json}
	Response: 
	\lstinputlisting[language=json]{./Iterazione 2/OtherFiles/ModifyAUserResponse.json}
	\caption{Documentazione API Modify a user.}
	\label{fig:ModifyAUserAPI}
\end{figure}

\begin{figure}[h!]
	\centering
	\includegraphics[width=1\linewidth]{./Iterazione 2/ImageFiles/GetOneTeam}
	\lstinputlisting[language=json]{./Iterazione 2/OtherFiles/GetOneTeamResponse.json}
	\caption{Documentazione API Get one team.}
	\label{fig:GetOneTeamAPI}
\end{figure}

\begin{figure}[h!]
	\centering
	\includegraphics[width=1\linewidth]{./Iterazione 2/ImageFiles/CreateATeam}
	Body:
	\lstinputlisting[language=json]{./Iterazione 2/OtherFiles/CreateATeamResponse.json}
	Response: 
	
	Team created successfully!
	\caption{Documentazione API Create a team.}
	\label{fig:CreateATeamAPI}
\end{figure}

\begin{figure}[h!]
	\centering
	\includegraphics[width=1\linewidth]{./Iterazione 2/ImageFiles/SetIdForemanRequest}
	
	Response: 
	
	Foreman 6 assigned to 3 correctly !
	\caption{Documentazione API Set id foreman.}
	\label{fig:SetIdForemanAPI}
\end{figure}


\begin{figure}[h!]
	\centering
	\includegraphics[width=1\linewidth]{./Iterazione 2/ImageFiles/GetAllAreas}
	\lstinputlisting[language=json]{./Iterazione 2/OtherFiles/GetAllAreasResponse.json}
	\caption{Documentazione API Get all areas.}
	\label{fig:GetAllAreasAPI}
\end{figure}

\clearpage

\includepdf[pages=-]{./Iterazione 2/OtherFiles/Report STAN4J/Quality Report}
