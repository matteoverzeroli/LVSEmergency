\section{Component Diagram}
Nella figura \ref{fig:ComponentDiagram_iterazione3} è riportato il diagramma a componenti dell'applicazione dove sono state evidenziate le interfacce e i componenti inseriti nell'iterazione 4. 
\begin{figure}[h!]
	\centering
	\includegraphics[width=1\linewidth]{./Iterazione 4/OtherFiles/UML - Component view V2}
	\caption{Component Diagram.}
	\label{fig:ComponentDiagram_iterazione3}
\end{figure}

\clearpage
\section{Class Diagram}
Di seguito è riportata la specifica della struttura dati \texttt{PositionDTO}(\Fig\ref{fig:ClassDiagramDTO_iterazione4}). Questo struttura è particolarmente utile per incapsulare in un unico oggetto sia le informazioni della posizione che il nome ed il cognome dell'utente a cui appartiene tale posizione. 

\begin{figure}[h!]
	\centering
	\includegraphics[width=1\linewidth]{./Iterazione 4/OtherFiles/DTOSpecification}
	\caption{Class Diagram \texttt{UserDTO} e \texttt{TeamDTO}}
	\label{fig:ClassDiagramDTO_iterazione4}
\end{figure}

\clearpage

\section{Interface and Package Diagram}

\begin{figure}[h!]
	\centering
	\includegraphics[width=0.8\linewidth]{./Iterazione 4/OtherFiles/UML - Interface Diagram}
	\caption{Interface and Package Diagram.}
\label{fig:InterfaceDiagram_iterazione3}
\end{figure}